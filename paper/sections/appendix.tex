% Appendix
\section{Prompts}
\label{sec:prompts}

\subsection{Text Analyst Prompt}
\label{sec:text_analyst_prompt}

\begin{verbatim}
You are analyzing reports for ZIP code {zip_code} during {start} to {end}.
You have access to: Tweets, 311 Calls, Sensor Readings.

Provide a structured assessment including:
1. Summary of relevant reports
2. Evidence for/against damage
3. Estimated damage percentage (0-100)
4. Confidence level (0-1)

Respond in JSON format.
\end{verbatim}

\subsection{Visual Analyst Prompt}
\label{sec:visual_analyst_prompt}

\begin{verbatim}
Analyze these satellite imagery tiles for ZIP code {zip_code}.
Assessment period: {start} to {end}.

Look for:
- Flooding extent
- Structural damage to buildings
- Road accessibility
- Debris or destruction

Provide damage estimate and confidence.
\end{verbatim}

% LLM Judge Prompts commented out - not used in current experiments
% \subsection{LLM Judge Prompts}
% \label{sec:judge_prompts}
% (Prompts for faithfulness and relevance evaluation are implemented but not included in current scope)


\section{Additional Results}
\label{sec:additional_results}

% Full Results Table (Appendix)
\begin{table}[h]
    \centering
    \caption{Complete results for all 100 evaluation queries (excerpt shown).}
    \label{tab:full_results}
    \resizebox{\textwidth}{!} & \textbf{Actual \%} & \textbf{Abs. Error} & \textbf{Faith.} & \textbf{Rel.} \\
            \midrule
            1                                    & 77479        & 08/26 - 09/01  & \todo{X.X}        & 0.18               & \todo{X.X}          & \todo{0.XX}     & \todo{0.XX}   \\
            2                                    & 77486        & 09/04 - 09/10  & \todo{X.X}        & 0.02               & \todo{X.X}          & \todo{0.XX}     & \todo{0.XX}   \\
            3                                    & 77058        & 08/26 - 09/01  & \todo{X.X}        & 1.58               & \todo{X.X}          & \todo{0.XX}     & \todo{0.XX}   \\
            \vdots                               & \vdots       & \vdots         & \vdots            & \vdots             & \vdots              & \vdots          & \vdots        \\
            25                                   & 77630        & 08/31 - 09/06  & \todo{X.X}        & 0.77               & \todo{X.X}          & \todo{0.XX}     & \todo{0.XX}   \\
            \midrule
            \multicolumn{3}{l}{\textbf{Average}} & ---          & ---            & \todo{X.XX}       & \todo{0.XX}        & \todo{0.XX}                                           \\
            \bottomrule
        \end{tabular}
    }
\end{table}


% Human Annotation Interface section removed - not used in current experiments


\section{Tweet Filtering Keywords}
\label{sec:tweet_keywords}

We implemented a keyword-based filtering pipeline to improve the signal-to-noise ratio of the 27 million raw tweets. A tweet is included if it contains at least one \textit{allow} keyword and no \textit{block} keywords.

\subsection{Allow List}
\label{sec:allow_list}

Keywords that indicate disaster-relevant content:

\begin{center}
\begin{tabular}{llll}
\toprule
flood & flooding & flooded & hurricane \\
storm & rain & underwater & rescue \\
trapped & stuck & help & emergency \\
911 & evacuate & damage & collapsed \\
power & outage & road & bridge \\
bayou & creek & & \\
\bottomrule
\end{tabular}
\end{center}

\subsection{Block List}
\label{sec:block_list}

During initial corpus exploration, we observed that many tweets containing disaster-related terms (e.g., ``Harvey'', ``Houston'') were unrelated to the hurricane. Common sources of noise included: (1) music promotion and streaming service spam using trending hashtags; (2) political commentary co-opting the disaster for unrelated messaging; (3) commercial advertisements and promotional giveaways; and (4) sports and entertainment discussions. The block list was developed iteratively by examining false positives in the filtered corpus.

Keywords that indicate irrelevant content:

\begin{center}
\begin{tabular}{llll}
\toprule
spotify & music & song & album \\
lyrics & vote & election & trump \\
biden & president & giveaway & contest \\
win & sale & shirt & merch \\
game & nfl & nba & football \\
baseball & love & heart & tears \\
\bottomrule
\end{tabular}
\end{center}

This filtering reduced the corpus from 27 million to approximately 450,000 tweets ($\sim$1.7\% acceptance rate).
