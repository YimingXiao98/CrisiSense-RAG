% Main LaTeX file for Multimodal RAG Paper
% Using NeurIPS 2024 style (can switch to ACL, AAAI, etc.)
\documentclass{article}

% Required packages
\usepackage[preprint]{neurips_2024}  % Change to [final] for camera-ready
\usepackage[utf8]{inputenc}
\usepackage[T1]{fontenc}
\usepackage{hyperref}
\usepackage{url}
\usepackage{booktabs}
\usepackage{amsfonts}
\usepackage{nicefrac}
\usepackage{microtype}
\usepackage{xcolor}
\usepackage{graphicx}
\usepackage{subcaption}
\usepackage{multirow}
\usepackage{array}
\usepackage{makecell}
\usepackage{amsmath}
\usepackage{amssymb}
\usepackage{algorithm}
\usepackage{algorithmic}

% Custom colors
\definecolor{modblue}{RGB}{31, 119, 180}
\definecolor{modgreen}{RGB}{44, 160, 44}
\definecolor{modred}{RGB}{214, 39, 40}

% Custom commands
\newcommand{\systemnameraw}{HarveyRAG}
\newcommand{\systemname}{\textsc{\systemnameraw}}
\newcommand{\todo}[1]{\textcolor{red}{[TODO: #1]}}
\newcommand{\fixme}[1]{\textcolor{orange}{[FIXME: #1]}}

\title{Multimodal Retrieval-Augmented Generation for \\
Disaster Impact Assessment: \\
A Case Study on Hurricane Harvey}

\author{
  Author One\thanks{Equal contribution} \\
  Department of Computer Science\\
  University Name\\
  \texttt{author1@university.edu} \\
  \And
  Author Two\footnotemark[1] \\
  Department of Computer Science\\
  University Name\\
  \texttt{author2@university.edu} \\
  \And
  Author Three \\
  Department Name\\
  Institution Name\\
  \texttt{author3@institution.edu} \\
}

\begin{document}

\maketitle

\begin{abstract}
    Accurate and timely disaster impact assessment is critical for effective emergency response and resource allocation. We present \systemname{}, a multimodal retrieval-augmented generation (RAG) system that integrates satellite imagery, social media posts, 311 emergency call records, and sensor data to assess flood impact during natural disasters. Our system employs a split-pipeline architecture that separates text and visual reasoning, enabling the use of specialized models for each modality. We evaluate \systemname{} on Hurricane Harvey (2017), comparing model predictions against FEMA flood depth measurements across 50 ZIP codes in the Greater Houston area. Our experiments demonstrate that multimodal context significantly reduces prediction error compared to text-only baselines, with spatial imagery providing critical calibration for text-based damage reports. We further analyze the contribution of different modalities and discuss implications for real-time disaster response systems.
\end{abstract}

% Include sections
% Introduction Section
\section{Introduction}
\label{sec:introduction}

\todo{Complete introduction section}

% Opening hook - the problem
Natural disasters cause billions of dollars in damage annually and require rapid, accurate impact assessment for effective emergency response. \cite{openai2023gpt4v} %TODO: add citation

% The challenge
Traditional disaster assessment methods rely on manual inspection, which is slow, dangerous, and resource-intensive. While satellite imagery provides broad coverage, interpreting visual data requires expertise and fails to capture ground-level conditions reported by affected residents. %TODO: add citation

% The opportunity
The proliferation of multimodal data sources—including high-resolution satellite imagery, social media, emergency call records, and IoT sensors—creates an unprecedented opportunity for automated, real-time disaster assessment. However, effectively integrating these heterogeneous data sources remains a significant challenge. %TODO: add citation

% Our approach
In this paper, we present \systemname{}, a multimodal retrieval-augmented generation (RAG) system that synthesizes satellite imagery, social media posts (Twitter/X), 311 emergency call records, and rainfall sensor data to assess structural damage following natural disasters. Our key insight is that different data modalities provide complementary evidence: satellite imagery reveals physical damage patterns, social media captures real-time resident experiences, and emergency calls indicate infrastructure failures.

% Technical contributions
Our technical contributions include:
\begin{enumerate}
    \item A \textbf{split-pipeline architecture} that separates text and visual reasoning, enabling specialized models for each modality and improved interpretability.
    \item A \textbf{hybrid retrieval system} combining dense embeddings, sparse keyword matching, and cross-modal reranking for effective multimodal context selection.
    \item A \textbf{systematic ablation study} comparing text-only, text-with-captions, and full multimodal configurations to quantify the contribution of each modality.
    \item An \textbf{extensive evaluation} on Hurricane Harvey (2017) using FEMA flood depth data as ground truth, demonstrating significant improvements over text-only baselines.
\end{enumerate}

% Paper organization
%TODO: uncomment when sections are finalized
%The remainder of this paper is organized as follows: Section~\ref{sec:related_work} reviews related work. Section~\ref{sec:methodology} describes our system architecture. Section~\ref{sec:experiments} details our experimental setup. Section~\ref{sec:results} presents quantitative results, and Section~\ref{sec:analysis} provides ablation studies and error analysis. We conclude in Section~\ref{sec:conclusion}.

% Related Work Section
\section{Related Work}
\label{sec:related_work}

\todo{Complete related work section}

\subsection{Disaster Impact Assessment}
\todo{Review traditional and ML-based disaster assessment methods}

% Placeholder for related work on disaster assessment
Prior work on disaster impact assessment can be broadly categorized into: (1) remote sensing approaches using satellite imagery, (2) social media analysis, and (3) multi-source data fusion. Our work differs by... %TODO: add citations

\subsection{Multimodal Retrieval-Augmented Generation}
\todo{Review RAG literature and multimodal extensions}

% Placeholder for RAG literature
Retrieval-augmented generation (RAG) has emerged as a powerful paradigm for grounding large language model outputs in external knowledge. Recent work has extended RAG to multimodal settings. However, existing approaches... %TODO: add citations

\subsection{Vision-Language Models for Document Understanding}
\todo{Review VLM literature relevant to our application}

% Placeholder for VLM literature
Vision-language models (VLMs) such as GPT-4V, Gemini, and LLaVA have demonstrated impressive capabilities in understanding images alongside text. In the context of disaster assessment... %TODO: add citations

% LLM-as-a-Judge section commented out - not used in current experiments
% \subsection{LLM-as-a-Judge Evaluation}
% Using large language models as evaluators has gained traction as a scalable 
% alternative to human evaluation. This is implemented in the codebase but not 
% included in the current experimental scope.

% Methodology Section
\section{Methodology}
\label{sec:methodology}

% System overview
We present \systemname{}, a multimodal RAG system for disaster impact assessment. %Figure~\ref{fig:system_architecture} provides an overview of our architecture. %TODO: add figure reference

\begin{figure}[t]
    \centering
    \includegraphics[width=0.95\textwidth]{figures/system_architecture.png}
    \caption{Overview of the \systemname{} architecture. Our system ingests multimodal data (satellite imagery, social media, 311 calls, sensors), retrieves relevant context, and employs a split-pipeline approach with separate text and visual analyzers before fusing results.}
    \label{fig:system_architecture}
\end{figure}

\subsection{Data Sources and Preprocessing}
\label{sec:data_sources}

Our system integrates four primary data sources for Hurricane Harvey (August-September 2017):

\paragraph{Satellite Imagery.}
We use high-resolution aerial imagery from NOAA's Emergency Response Imagery \todo{cite}. The dataset contains approximately 3,500 tiles covering the Greater Houston metropolitan area at 0.5m resolution. Images are indexed by geographic coordinates and acquisition date.

\paragraph{Visual-Text Bridging.}
To enable semantic retrieval of visual data, we leveraged \textbf{Gemini 2.0 Flash Verified} to generate descriptive captions for all \textbf{3,507 aerial image tiles}. These captions are indexed as searchable text documents, allowing the RAG system to retrieve imagery based on textual descriptions of damage (e.g., ``submerged houses'').

\paragraph{Social Media (Twitter/X).}
We implemented a strict filtering pipeline for the 27 million raw tweets collected during Hurricane Harvey to improve signal-to-noise ratio. The process involved: (1) \textbf{De-Duplication}: Removed retweets (approx. 57\% of data); (2) \textbf{Content Filters}: Applied an \textit{Allow List} (e.g., ``flood'', ``rescue'') and a \textit{Block List} (e.g., ``spotify'', ``music'') to remove irrelevant chatter; and (3) \textbf{Spam Removal}: Filtered posts with excessive hashtags/URLs. This reduced the corpus to $\sim$450,000 highly relevant unique tweets ($\sim$1.7\% acceptance rate).

\paragraph{311 Emergency Calls.}
We obtain records from the City of Houston's 311 service system, including reports of flooding, debris, and infrastructure damage. The dataset contains 26,107 records from August 20--September 10, 2017, covering 125 unique ZIP codes in the urban Houston area.


\paragraph{Rainfall Sensors.}
We incorporate data from Harris County Flood Control District rain gauges. To provide localized context, we map each target ZIP code to the nearest available sensor using a centroid-based nearest neighbor lookup, retrieving hourly precipitation measurements for the relevant time window.

\subsection{Multimodal Retrieval}
\label{sec:retrieval}

Given a query specifying a ZIP code and time window, our retrieval system identifies the most relevant context from each data source.

\paragraph{Text Retrieval.}
We employ a hybrid retrieval strategy that combines dense and sparse methods with a reranking stage. For dense retrieval, we generate text embeddings using Sentence-BERT to capture semantic similarity. This is complemented by BM25-based sparse retrieval for keyword matching. Finally, a cross-encoder reranker refines the top candidates to optimize precision.

\paragraph{Visual Retrieval.}
For satellite imagery, searches first attempt to find tiles strictly within the target ZIP code polygon. If insufficient imagery is found, the system falls back to a spatial radius search (e.g., 5km) to retrieve relevant nearby tiles from the same acquisition period.

\subsection{Split-Pipeline Architecture}
\label{sec:split_pipeline}

A key design decision in \systemname{} is the separation of text and visual reasoning into distinct analysis pipelines.

\begin{figure}[t]
    \centering
    \includegraphics[width=0.9\textwidth]{figures/split_pipeline.png}
    \caption{Split-pipeline architecture with Visual Additive Fusion. Text analysis serves as the primary signal, while visual analysis provides additive confirmation.}
    \label{fig:split_pipeline}
\end{figure}

\paragraph{Text Analyst.}
The Text Analyst module processes retrieved text snippets---including tweets, 311 calls, and sensor readings---and generates a structured analysis, serving as the primary evidence source for local impact.

\paragraph{Visual Analyst.}
The Visual Analyst module processes retrieved satellite imagery tiles to produce a comprehensive visual assessment of flood extent and structural damage.

\paragraph{Fusion Engine (Visual Additive).}
We employ a \textit{Visual Additive} fusion strategy designed to robustly handle temporal misalignments between peak-flood text reports and post-flood snapshot imagery. Text evidence ($E_{text}$) serves as the primary signal for flood extent (Hazard). Visual evidence ($E_{visual}$) is strictly additive and prioritized for structural damage (Consequence):
\begin{equation}
    Score_{final} = \text{Norm}(Score(E_{text}) + \lambda \cdot \mathbb{I}(E_{visual} \text{ confirms damage}))
\end{equation}
This design addresses the "snapshot vs. peak" problem: a clean satellite image taken days after the peak flood should not veto valid text reports of flooding (which occurred earlier), but visible structural damage (debris, blue tarps) serves as a lasting and reliable confirmation signal.

% NOTE: LLM-as-a-Judge evaluation is implemented but not used in current experiments.
% This section is commented out for now and can be re-enabled for future work.

% \subsection{LLM-as-a-Judge Evaluation}
% \label{sec:llm_judge}
% 
% To evaluate response quality beyond quantitative accuracy, we employ multiple LLM judges to assess:
% 
% \paragraph{Faithfulness.}
% Whether claims in the response are supported by the retrieved context:
% \begin{equation}
%     \text{Faithfulness} = \frac{|\text{Supported Claims}|}{|\text{All Claims}|}
% \end{equation}
% 
% \paragraph{Relevance.}
% Whether the response addresses the query:
% \begin{equation}
%     \text{Relevance} = \text{Judge}(Q, A) \in [0, 1]
% \end{equation}
% 
% % LLM Judges Table
\begin{table}[t]
    \centering
    \caption{LLM judges used for response quality evaluation.}
    \label{tab:judges}
    \begin{tabular}{llcl}
        \toprule
        \textbf{Judge}   & \textbf{Provider} & \textbf{Parameters} & \textbf{Notes}         \\
        \midrule
        GPT-4o           & OpenAI            & N/A                 & Production model       \\
        GPT-4o-mini      & OpenAI            & N/A                 & Cost-efficient variant \\
        Gemini 2.5 Flash & Google            & N/A                 & Multimodal-native      \\
        Gemma 3-27B      & HuggingFace       & 27B                 & Open-source            \\
        Qwen3-VL-235B    & HuggingFace       & 235B                & Open-source, large     \\
        \bottomrule
    \end{tabular}
\end{table}


% Experiments Section
\section{Experiments}
\label{sec:experiments}

\subsection{Dataset}
\label{sec:dataset}

We evaluate \systemname{} on data from Hurricane Harvey, which made landfall on August 25, 2017, causing unprecedented flooding in the Greater Houston area. %Table~\ref{tab:dataset_stats} summarizes our dataset. %TODO: add table reference

% Dataset Statistics Table
\begin{table}[t]
    \centering
    \caption{Dataset statistics for the Hurricane Harvey evaluation.}
    \label{tab:dataset_stats}
    \begin{tabular}{lrr}
        \toprule
        \textbf{Data Source}            & \textbf{Count} & \textbf{Coverage} \\
        \midrule
        Satellite Imagery Tiles         & 3,507          & Greater Houston   \\
        Image Captions (generated)      & 3,507          & Greater Houston   \\
        Tweets (filtered)               & 450,000        & Harris County     \\
        311 Emergency Calls             & 26,107         & City of Houston   \\
        Rainfall Sensors                & 150+           & Harris County     \\
        Point Damage Estimates (PDE)    & 72,755         & 139 ZIPs          \\
        \midrule
        ZIP Codes (evaluation)          & 50             & Study Area        \\
        Flood Depth Grid (ground truth) & 33,144 ZIPs    & Texas-wide        \\
        \bottomrule
    \end{tabular}
\end{table}



\paragraph{Dual Ground Truth.}
We employ a dual ground-truth framework to evaluate the distinct "Hazard" and "Consequence" outputs of our system:

\begin{itemize}
    \item \textbf{Hazard Ground Truth (Flood Extent)}: We use the FEMA Harvey Flood Depth Grid to compute the percentage of each ZIP code covered by water (\texttt{flooded\_pct}). This serves as the target for our model's \texttt{flood\_extent\_pct} prediction.
    \item \textbf{Damage Ground Truth (Structural Severity)}: We utilize Point Damage Estimates (PDE) and FEMA NFIP Claims to quantify structural impact. The PDE score (normalized 0--100) serves as the primary target for our model's \texttt{damage\_severity\_pct} prediction, representing the intensity of physical destruction rather than just water presence.
\end{itemize}

\paragraph{Query Construction.}
We construct 50 evaluation queries with a stratified sampling design to ensure diverse geographic and data coverage. The evaluation set comprises 35 queries from ZIP codes with high 311 call volume (representing urban Houston with dense ground-truth signals) and 15 queries from suburban/peripheral ZIP codes. All queries target the Hurricane Harvey impact period (August 25--September 10, 2017) with 7-day rolling windows.

\subsection{Ablation Study Design}
\label{sec:baselines}

We conduct a systematic ablation study to quantify the contribution of each modality:

\paragraph{Text-Only RAG.}
Baseline RAG using only text sources (tweets, 311 calls, sensors) without any imagery input. This represents a purely text-based assessment approach.

\paragraph{Text + Caption RAG.}
RAG with text sources augmented by image captions. Captions are generated from satellite imagery using Gemini 2.0 Flash and indexed as searchable text. This tests whether semantic descriptions of imagery can bridge the modality gap without direct visual analysis.

\paragraph{Full Multimodal RAG (\systemname{}).}
Complete split-pipeline architecture with separate Text Analyst and Visual Analyst modules. The Visual Analyst directly processes satellite imagery tiles using a vision-language model, providing independent visual assessment that is fused with text analysis.

\paragraph{Temporal Context for Caption Fusion.}
A critical challenge in early fusion via captioning is the temporal mismatch between text sources (reporting real-time conditions during peak flooding) and satellite imagery captions (describing post-flood scenes). We address this through explicit prompt engineering that provides temporal context:
\begin{itemize}
    \item \textbf{For Flood Extent (Hazard)}: The LLM is instructed to \textit{ignore} caption statements about flooding (e.g., ``no visible flooding'') and trust text sources, since water recedes before image capture.
    \item \textbf{For Damage Severity (Consequence)}: The LLM is instructed to \textit{use} caption damage indicators (debris, structural damage, discoloration) as confirmatory evidence that boosts confidence in text-reported damage.
\end{itemize}
This interpretation guidance resolves the apparent contradiction between modalities, enabling captions to provide complementary value rather than introducing noise.

\subsection{Models}
\label{sec:models}

We evaluate multiple model configurations:

% Models Table
\begin{table}[t]
    \centering
    \caption{Model configurations evaluated.}
    \label{tab:models}
    \begin{tabular}{lllc}
        \toprule
        \textbf{Configuration} & \textbf{Text Model} & \textbf{Visual Model} & \textbf{Fusion} \\
        \midrule
        \multicolumn{4}{l}{\textit{Baselines}}                                                 \\
        Text-Only RAG          & Gemini 2.5 Flash    & ---                   & ---             \\
        Visual-Only            & ---                 & GPT-4o                & ---             \\
        Unified                & Gemini 2.5 Flash    & (same)                & Implicit        \\
        \midrule
        \multicolumn{4}{l}{\textit{\systemname{} Variants}}                                    \\
        Split (Gemini+Gemini)  & Gemini 2.5 Flash    & Gemini 2.5 Flash      & GPT-4o          \\
        Split (Gemini+GPT4o)   & Gemini 2.5 Flash    & GPT-4o                & GPT-4o          \\
        \bottomrule
    \end{tabular}
\end{table}


\subsection{Evaluation Metrics}
\label{sec:metrics}

\paragraph{Quantitative Metrics.}
We evaluate prediction accuracy using two complementary metrics: Mean Absolute Error (MAE) measures the average deviation between predicted flood impact and actual flood depth in meters; Spearman's $\rho$ assesses rank correlation with ground truth, indicating whether the system correctly orders regions by flood severity.

% NOTE: Quality metrics (Faithfulness, Relevance) and Human Evaluation are implemented
% in the codebase but not included in the current experimental scope.
% 
% \paragraph{Quality Metrics.}
% Beyond quantitative accuracy, we assess response quality through Faithfulness 
% (proportion of claims supported by retrieved context) and Relevance (query-response alignment).
% 
% \subsection{Human Evaluation}
% To validate LLM judge scores, we conduct human annotation on selected queries.

% Results Section
\section{Results}
\label{sec:results}

\subsection{Main Results}
\label{sec:main_results}

Table~\ref{tab:main_results} presents the main quantitative results from our ablation study comparing three configurations: text-only, text with captions, and full multimodal RAG.

\begin{table}[t]
    \centering
    \caption{Ablation study on 15-query validation set using the dual-output schema. We report Mean Absolute Error (MAE) for both Hazard (Flood Extent) and Consequence (Structural Damage). With proper temporal context in prompts, Text+Caption now outperforms Text-Only on both metrics, demonstrating that image captions provide valuable signals when interpreted correctly.}
    \label{tab:main_results}
    \begin{tabular}{lcc}
        \toprule
        \textbf{Method}                 & \textbf{Hazard MAE (\%)} $\downarrow$ & \textbf{Damage MAE (\%)} $\downarrow$ \\
        \midrule
        Text-Only RAG                   & 22.70                                 & 29.25                                 \\
        Text + Caption RAG              & \textbf{20.16}                        & \textbf{26.39}                        \\
        \bottomrule
    \end{tabular}
\end{table}



\paragraph{Key Findings.}
Our results demonstrate that image captions, when properly interpreted, significantly improve disaster assessment:

\textbf{(1) Text+Caption Achieves Best Performance.} With proper temporal context instructions, the Text+Caption pipeline achieves the lowest error on both metrics: \textbf{20.16\%} MAE for Flood Extent and \textbf{26.39\%} MAE for Damage Severity. This represents a 2.54 and 2.86 percentage point improvement over Text-Only, respectively.

\textbf{(2) Temporal Context is Critical.} The key to successful caption fusion is teaching the LLM that satellite imagery was captured \textit{after} peak flooding. Without this context, captions like ``no visible flooding'' contradict text evidence and degrade performance. With proper instructions, the LLM correctly interprets these as ``water receded'' rather than ``flooding didn't occur.''

\textbf{(3) Captions Provide Complementary Signals.} Image captions offer damage indicators (debris, structural damage, discoloration) that confirm and contextualize text reports, while text sources provide real-time flood extent information that imagery cannot capture due to temporal mismatch.

\subsection{Per-Category Performance}
\label{sec:category_results}

We analyze performance across different flood severity levels to understand where multimodal context provides the greatest benefit.

\begin{figure}[t]
    \centering
    % Placeholder for category performance bar chart
    \fbox{\parbox{0.9\textwidth}{\centering\vspace{2cm}
            \textbf{[PLACEHOLDER: Bar Chart]}\\
            Show: MAE by flood depth category (High/Medium/Low) for each model
            \vspace{2cm}}}
    \caption{Performance by flood severity. \systemname{} shows consistent improvements across all damage levels, with the largest gains in high-severity regions where visual confirmation is most critical.}
    \label{fig:category_performance}
\end{figure}

% Category Results Table
\begin{table}[t]
    \centering
    \caption{Performance breakdown by damage severity category.}
    \label{tab:category_results}
    \begin{tabular}{lccc|ccc}
        \toprule
                             & \multicolumn{3}{c|}{\textbf{MAE} $\downarrow$} & \multicolumn{3}{c}{\textbf{RMSE} $\downarrow$}                                                      \\
        \textbf{Method}      & High                                           & Low                                            & Zero       & High        & Low        & Zero       \\
        \midrule
        Text-Only RAG        & \todo{XX.X}                                    & \todo{X.X}                                     & \todo{X.X} & \todo{XX.X} & \todo{X.X} & \todo{X.X} \\
        Visual-Only          & \todo{XX.X}                                    & \todo{X.X}                                     & \todo{X.X} & \todo{XX.X} & \todo{X.X} & \todo{X.X} \\
        Unified Pipeline     & \todo{XX.X}                                    & \todo{X.X}                                     & \todo{X.X} & \todo{XX.X} & \todo{X.X} & \todo{X.X} \\
        \midrule
        \systemname{} (Ours) & \todo{XX.X}                                    & \todo{X.X}                                     & \todo{X.X} & \todo{XX.X} & \todo{X.X} & \todo{X.X} \\
        \bottomrule
    \end{tabular}
\end{table}


\subsection{Ablation Analysis}
\label{sec:ablation}

\paragraph{Impact of Temporal Context.}
The success of our Text+Caption approach hinges on explicit temporal context instructions. Without guidance, the LLM sees conflicting evidence: tweets report ``house flooded'' while captions state ``no visible flooding.'' By explaining that imagery was captured days after peak flooding, the LLM correctly resolves this conflict---trusting text for flood extent while using caption damage indicators as confirmatory evidence.

\paragraph{Role of Image Captions.}
With proper interpretation guidance, image captions provide valuable complementary signals. Captions excel at describing persistent damage (debris fields, structural collapse, vegetation damage) that remains visible in post-flood imagery. This information confirms and contextualizes text reports, reducing overall error by 2.5--2.9 percentage points compared to text-only approaches.

% NOTE: LLM-as-a-Judge and Human-AI Agreement sections are commented out as these
% evaluations are not included in the current experimental scope. They are implemented
% in the codebase and can be enabled for future work.
%
% \subsection{LLM Judge Evaluation}
% \subsection{Human-AI Agreement}

% Analysis Section
\section{Analysis}
\label{sec:analysis}

\subsection{Ablation Study}
\label{sec:ablation_analysis}

We conduct ablation experiments to understand the contribution of each component.

% Ablation Study Table
\begin{table}[t]
    \centering
    \caption{Ablation study results. $\Delta$ indicates change from full model.}
    \label{tab:ablation}
    \begin{tabular}{lccc}
        \toprule
        \textbf{Configuration}    & \textbf{MAE} & \textbf{$\Delta$ MAE} & \textbf{Faithfulness} \\
        \midrule
        \multicolumn{4}{l}{\textit{Full Model}}                                                  \\
        \systemname{} (Full)      & \todo{X.XX}  & ---                   & \todo{0.XX}           \\
        \midrule
        \multicolumn{4}{l}{\textit{Modality Ablation}}                                           \\
        \quad - Satellite Imagery & \todo{XX.XX} & +\todo{XX.X}          & \todo{0.XX}           \\
        \quad - Tweets            & \todo{XX.XX} & +\todo{X.X}           & \todo{0.XX}           \\
        \quad - 311 Calls         & \todo{XX.XX} & +\todo{X.X}           & \todo{0.XX}           \\
        \quad - Sensor Data       & \todo{XX.XX} & +\todo{X.X}           & \todo{0.XX}           \\
        \midrule
        \multicolumn{4}{l}{\textit{Architecture Ablation}}                                       \\
        \quad Without Retrieval   & \todo{XX.XX} & +\todo{XX.X}          & \todo{0.XX}           \\
        \quad Without Reranking   & \todo{XX.XX} & +\todo{X.X}           & \todo{0.XX}           \\
        \quad Without Fusion      & \todo{XX.XX} & +\todo{X.X}           & \todo{0.XX}           \\
        \bottomrule
    \end{tabular}
\end{table}


\paragraph{Modality Ablation.}
Our dual-metric evaluation (Table~\ref{tab:main_results}) reveals that proper handling of image captions significantly improves performance over text-only approaches.

\textbf{Caption Success with Temporal Context}: Our Text+Caption pipeline achieves the best performance on both metrics: \textbf{20.16\%} MAE for Flood Extent (a 2.54 percentage point improvement over Text-Only) and \textbf{26.39\%} MAE for Damage Severity (a 2.86 percentage point improvement). This success stems from explicit temporal context instructions that teach the LLM to correctly interpret satellite imagery captions:
\begin{itemize}
    \item Captions describe imagery captured on August 31st, \textit{after} peak flooding (August 27--28).
    \item ``No visible flooding'' means water \textit{receded}, not that flooding didn't occur---the LLM is instructed to trust text for flood extent.
    \item Damage indicators (debris, structural damage) in captions \textit{confirm} text reports, boosting confidence.
\end{itemize}

\textbf{Importance of Prompt Engineering}: Without proper temporal context, naive caption fusion \textit{degraded} performance in prior experiments (MAE increased by 2+ points). The key insight is that captions are not inherently noisy---they require interpretation guidance to resolve the apparent contradiction between post-flood imagery and real-time text reports.

\paragraph{Architecture Ablation.}
We also evaluate architectural choices. Direct prompting without RAG significantly degrades performance, underscoring the importance of retrieval-augmented context. Removing the cross-encoder reranker reduces response quality, indicating that refined candidate selection contributes meaningfully to accuracy. Finally, we find that simply averaging text and visual estimates underperforms our learned fusion approach, suggesting that modality integration benefits from trainable parameters.

\subsection{Geographic Boosting Analysis}
\label{sec:geo_boosting}

We investigated whether prioritizing documents strictly from the query's ZIP code would improve accuracy (Exp2). Our hypothesis was that local context is most predictive. However, results were \textbf{negative}: MAE increased from 3.07\% (Baseline) to 4.94\% (Geo-Enhanced).

Analysis revealed that boosting local documents often surfaced irrelevant reports (e.g., ``blocked driveway'') over highly relevant but distant descriptions (e.g., ``major flooding nearby''). This finding suggests that \textbf{semantic relevance is more critical than strict geographic proximity} for disaster impact assessment.

\subsection{Error Analysis}
\label{sec:error_analysis}

%Figure~\ref{fig:error_cases} shows representative error cases. %TODO: add figure reference

\begin{figure}[t]
    \centering
    % Placeholder for error case examples
    \fbox{\parbox{0.9\textwidth}{\centering\vspace{2cm}
            \textbf{[PLACEHOLDER: Error Case Examples]}\\
            Show: 2-3 failure cases with analysis
            \vspace{2cm}}}
    \caption{Representative error cases. Top: \todo{description}. Bottom: \todo{description}.}
    \label{fig:error_cases}
\end{figure}

\paragraph{Common Error Types.}
Our analysis reveals three primary categories of errors. First, context gaps occur in ZIP codes with sparse data coverage, where insufficient evidence leads to unreliable predictions. Second, temporal misalignment arises when the system confuses pre-storm and post-storm imagery, resulting in incorrect damage assessments. Third, false positives stem from misinterpreting normal water features such as swimming pools or retention ponds as flood damage.

\subsection{Retrieval Quality}
\label{sec:retrieval_analysis}

We analyze retrieval effectiveness using standard IR metrics.

% Retrieval Metrics Table
\begin{table}[t]
    \centering
    \caption{Retrieval quality metrics.}
    \label{tab:retrieval_metrics}
    \begin{tabular}{lccc}
        \toprule
        \textbf{Modality}   & \textbf{Precision@10} & \textbf{Recall@10} & \textbf{MRR} \\
        \midrule
        Text (Tweets + 311) & \todo{0.XX}           & \todo{0.XX}        & \todo{0.XX}  \\
        Visual (Imagery)    & \todo{0.XX}           & \todo{0.XX}        & \todo{0.XX}  \\
        Combined            & \todo{0.XX}           & \todo{0.XX}        & \todo{0.XX}  \\
        \bottomrule
    \end{tabular}
\end{table}


\subsection{Qualitative Examples}
\label{sec:qualitative}

%Figure~\ref{fig:qualitative} shows example system outputs. %TODO: add figure reference

\begin{figure}[t]
    \centering
    % Placeholder for qualitative examples
    \fbox{\parbox{0.9\textwidth}{\centering\vspace{3cm}
            \textbf{[PLACEHOLDER: Qualitative Examples]}\\
            Show: Query + Retrieved Context + Model Response + Ground Truth\\
            For 2-3 representative cases
            \vspace{3cm}}}
    \caption{Example system outputs showing retrieved context and generated assessments.}
    \label{fig:qualitative}
\end{figure}

% Conclusion Section
\section{Conclusion}
\label{sec:conclusion}

We presented \systemname{}, a multimodal retrieval-augmented generation system for disaster impact assessment. Our work demonstrates that integrating satellite imagery with text-based sources (social media, 311 emergency calls) significantly improves flood impact prediction accuracy. Through a systematic ablation study on Hurricane Harvey data, we show that full multimodal RAG achieves \todo{XX\%} lower error compared to text-only baselines, with visual analysis providing critical calibration for ground-level reports.

\paragraph{Limitations.}
Our evaluation is limited to a single disaster event (Hurricane Harvey), which may constrain generalizability to other disaster types or geographic regions. Ground truth based on flood depth measurements may not fully capture all aspects of infrastructure damage. Additionally, our social media analysis focuses on English-language content, potentially missing perspectives from non-English speaking communities.

\paragraph{Future Work.}
We plan to extend this work by evaluating across multiple disaster types and geographic regions. Integration of additional data sources such as news articles and government damage reports could provide richer context. Development of real-time assessment capabilities for active disaster events represents an important direction for operational deployment.

\paragraph{Broader Impact.}
Our system has potential to assist emergency responders in rapid damage assessment. However, we caution against using automated systems as the sole basis for resource allocation decisions, as errors could disproportionately affect vulnerable communities.


\bibliography{references}
\bibliographystyle{plainnat}

\appendix
% Appendix
\section{Prompts}
\label{sec:prompts}

\subsection{Text Analyst Prompt}
\label{sec:text_analyst_prompt}

\begin{verbatim}
You are analyzing reports for ZIP code {zip_code} during {start} to {end}.
You have access to: Tweets, 311 Calls, Sensor Readings.

Provide a structured assessment including:
1. Summary of relevant reports
2. Evidence for/against damage
3. Estimated damage percentage (0-100)
4. Confidence level (0-1)

Respond in JSON format.
\end{verbatim}

\subsection{Visual Analyst Prompt}
\label{sec:visual_analyst_prompt}

\begin{verbatim}
Analyze these satellite imagery tiles for ZIP code {zip_code}.
Assessment period: {start} to {end}.

Look for:
- Flooding extent
- Structural damage to buildings
- Road accessibility
- Debris or destruction

Provide damage estimate and confidence.
\end{verbatim}

% LLM Judge Prompts commented out - not used in current experiments
% \subsection{LLM Judge Prompts}
% \label{sec:judge_prompts}
% (Prompts for faithfulness and relevance evaluation are implemented but not included in current scope)


\section{Additional Results}
\label{sec:additional_results}

% Full Results Table (Appendix)
\begin{table}[h]
    \centering
    \caption{Complete results for all 100 evaluation queries (excerpt shown).}
    \label{tab:full_results}
    \resizebox{\textwidth}{!} & \textbf{Actual \%} & \textbf{Abs. Error} & \textbf{Faith.} & \textbf{Rel.} \\
            \midrule
            1                                    & 77479        & 08/26 - 09/01  & \todo{X.X}        & 0.18               & \todo{X.X}          & \todo{0.XX}     & \todo{0.XX}   \\
            2                                    & 77486        & 09/04 - 09/10  & \todo{X.X}        & 0.02               & \todo{X.X}          & \todo{0.XX}     & \todo{0.XX}   \\
            3                                    & 77058        & 08/26 - 09/01  & \todo{X.X}        & 1.58               & \todo{X.X}          & \todo{0.XX}     & \todo{0.XX}   \\
            \vdots                               & \vdots       & \vdots         & \vdots            & \vdots             & \vdots              & \vdots          & \vdots        \\
            25                                   & 77630        & 08/31 - 09/06  & \todo{X.X}        & 0.77               & \todo{X.X}          & \todo{0.XX}     & \todo{0.XX}   \\
            \midrule
            \multicolumn{3}{l}{\textbf{Average}} & ---          & ---            & \todo{X.XX}       & \todo{0.XX}        & \todo{0.XX}                                           \\
            \bottomrule
        \end{tabular}
    }
\end{table}


% Human Annotation Interface section removed - not used in current experiments


\section{Tweet Filtering Keywords}
\label{sec:tweet_keywords}

We implemented a keyword-based filtering pipeline to improve the signal-to-noise ratio of the 27 million raw tweets. A tweet is included if it contains at least one \textit{allow} keyword and no \textit{block} keywords.

\subsection{Allow List}
\label{sec:allow_list}

Keywords that indicate disaster-relevant content:

\begin{center}
\begin{tabular}{llll}
\toprule
flood & flooding & flooded & hurricane \\
storm & rain & underwater & rescue \\
trapped & stuck & help & emergency \\
911 & evacuate & damage & collapsed \\
power & outage & road & bridge \\
bayou & creek & & \\
\bottomrule
\end{tabular}
\end{center}

\subsection{Block List}
\label{sec:block_list}

During initial corpus exploration, we observed that many tweets containing disaster-related terms (e.g., ``Harvey'', ``Houston'') were unrelated to the hurricane. Common sources of noise included: (1) music promotion and streaming service spam using trending hashtags; (2) political commentary co-opting the disaster for unrelated messaging; (3) commercial advertisements and promotional giveaways; and (4) sports and entertainment discussions. The block list was developed iteratively by examining false positives in the filtered corpus.

Keywords that indicate irrelevant content:

\begin{center}
\begin{tabular}{llll}
\toprule
spotify & music & song & album \\
lyrics & vote & election & trump \\
biden & president & giveaway & contest \\
win & sale & shirt & merch \\
game & nfl & nba & football \\
baseball & love & heart & tears \\
\bottomrule
\end{tabular}
\end{center}

This filtering reduced the corpus from 27 million to approximately 450,000 tweets ($\sim$1.7\% acceptance rate).


\end{document}
